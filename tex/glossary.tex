\usepackage[sort=none,abbreviations]{glossaries-extra}
\setabbreviationstyle{long-short}

\newglossaryentry{corrutina}
{
	name=Corrutina,
    text=corrutina,
	description={Una unidad de procesamiento que puede ejecutarse asincronicamente. Pueden o no ejecutarse multiples corrutinas en paralelo.}
}

\newglossaryentry{esc}
{
	name=Electronic Speed Control,
	text=ESC,
	description={Controlador de velocidad para motores eléctricos sin escobillas. Suelen tener DC a la entrada y trifásica a la salida.}
}

\newglossaryentry{cad}
{
	name=Computer Aided Design,
	text=CAD,
	description={El uso de computadoras para la creación, modificación, analisis o optimización de un diseño. Es comunmente usado para referirse a un documento digital que contiene información persistente de un diseño.}
}

\newglossaryentry{edf}
{
	name=Electronic ducted fan,
	text=EDF,
	description={Turbina impulsada por un motor eléctrico brushless trifásico.}
}

\newglossaryentry{lqr}
{
	name=Linear Quadratic Regulator,
	text=LQR,
	description={Regulador basado en control óptimo que busca reducir error cuadratico de una función costo.}
}

\newglossaryentry{soc}
{
	name=System on a chip,
	text=SoC,
	description={Integración de modulos conectados a un controlador en un único circuito impreso.}
}

\newglossaryentry{datarace}
{
	name=Data race,
	text=data race,
	description={Situación en un programa concurrente donde un proceso puede acceder/escribir una ubicación de memoria al mismo tiempo que otro proceso escribe esa memoria.}
}

\newglossaryentry{lia}
{
	name=LIA Aerospace,
	text=LIA Aerospace,
	description={Laboratorio de Investigaciónes Aeroespaciales. Empresa para la cual se efectuó el desarrollo del vehículo.}
}

\newabbreviation{i2c}
{I$^2$C}
{Protocolo de comunicación de dos hilos que permite comunicar varios circuitos integrados en un bus.}

\newabbreviation{gpio}
{GPIO}
{Salida digital de uso genérico. Pueden estar en \emph{high} (tensión de fuente) o \emph{low} (puesto a tierra).}

\newabbreviation{spi}
{SPI}
{Serial peripheral interface. Un protocolo de comunicación full-duplex de Motorola.}

\newabbreviation{uart}
{UART}
{Protocolo de comunicación universal asincrono.}

\newabbreviation{uav}
{UAV}
{Vehículo aéreo no tripulado, del inglés \textit{Unmanned aerial vehicle}.}

\newabbreviation{vtvl}
{VTVL}
{Es un tipo de vehículo que tiene la capacidad de despegar y aterrizar verticalmente, del inglés \textit{Vertical take-off, vertical landing}.}
\newabbreviation{tvc}
{TVC}
{El \"Thrust Vector Control\" (TVC), en español \"Control de Vector de Empuje\", es un sistema utilizado en vehículos propulsados por un vector de un eje (como cohetes o misiles) para controlar y dirigir la trayectoria del vehículo.}


% \newabbreviation{esc}
% {ESC}
% {Electronic speed control.}

% \newabbreviation{edf}
% {EDF}
% {Electronic ducted fan.}

\usepackage[sort=none,abbreviations]{glossaries-extra}
\setabbreviationstyle{long-short}

\newglossaryentry{corrutina}
{
	name=Corrutina,
    text=corrutina,
	description={Una unidad de procesamiento que puede ejecutarse en simultaneo con otras corrutinas.}
}

\newglossaryentry{esc}
{
	name=Electronic Speed Control,
	text=ESC,
	description={Controlador de velocidad para motores brushless. Suelen tener DC a la entrada y trifásica a la salida.}
}

\newglossaryentry{edf}
{
	name=Electronic ducted fan,
	text=EDF,
	description={Turbina impulsada por un motor eléctrico brushless. Es eficiente a altas velocidades.}
}

\newglossaryentry{lqr}
{
	name=Linear Quadratic Regulator,
	text=LQR,
	description={Regulador basado en control óptimo que busca reducir error cuadratico de una función costo.}
}

\newglossaryentry{soc}
{
	name=System on a chip,
	text=SoC,
	description={Integración de modulos conectados a un controlador en un único circuito impreso.}
}

\newglossaryentry{datarace}
{
	name=Data race,
	text=data race,
	description={Situación en un programa concurrente donde un proceso puede acceder/escribir una ubicación de memoria al mismo tiempo que otro proceso escribe esa memoria.}
}

\newglossaryentry{lia}
{
	name=LIA,
	text=LIA Aerospace,
	description={Laboratorio de Investigaciónes Aeroespaciales. Empresa para la cual se efectuó el desarrollo del vehículo.}
}

\newabbreviation{i2c}
{I$^2$C}
{Protocolo de comunicación de dos hilos que permite comunicar varios circuitos integrados en un bus.}

\newabbreviation{gpio}
{GPIO}
{Salida digital de uso genérico. Pueden estar en \emph{high} (tensión de fuente) o \emph{low} (puesto a tierra).}

\newabbreviation{spi}
{SPI}
{Serial peripheral interface. Un protocolo de comunicación full-duplex de Motorola.}

\newabbreviation{uart}
{UART}
{Protocolo de comunicación universal asincrono.}

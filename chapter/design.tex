\section{Diseño}

En el marco de este proyecto, es de interés el diseño mecánico del mecanismo de control semejante un vehículo propulsado por combustión externa. Estos últimos suelen ser dirigidos por toberas montadas sobre un cardán. En el caso de un vehículo eléctrico de una hélice se tendría que montar el sistema de propulsión centrado en un gimbal para no obstruir el flujo. Se decidió montar el EDF sobre un gimbal para simular el control que tienen los vehículos VTVL a reacción de la industria aeroespacial. Al no usar álabes para controlar el flujo saliente, como es usual en los vehículos VTVL monorrotor se evita el fenómeno de desprendimiento de capa límite para ángulos de ataque altos.

\subsection{Selección de propulsor}

Los requerimientos para el propulsor son los siguientes

\begin{itemize}
    \item Empuje que mantenga altura del vehículo para la máxima actuación del gimbal
    \item Disponible comercialmente
    \item Precio de propulsor y accesorios relevantes dentro de presupuesto
    \item Durabilidad ante pruebas físicas en cuanto a la duración del proyecto
\end{itemize}

Estos requerimientos fueron los primeros especificados para el diseño del vehículo. El diseño estructural y la selección de componentes electrónicos parten del propulsor seleccionado ya que este es la pieza central para lograr los objetivos propuestos y la que más puede variar en especificaciones técnicas.

\medskip 

Existen diversas maneras de impulsar un vehículo de forma eléctrica. Luego de un cuidadoso estudio y discusión de ingeniería se decidió optar por un \gls{edf}, quien juega un rol central en el diseño pues es lo que se intenta controlar para llevar a cabo la navegación y guiado.


\medskip

Se generó una lista de EDF's de diferentes diámetros disponibles comercialmente.

\begin{table}[!ht]
    \centering
    \begin{tabular}{l|r|r|r}
Diámetro de EDF    & \multicolumn{1}{l|}{70mm} & \multicolumn{1}{l|}{90mm} & \multicolumn{1}{l}{120mm} \\ \hline
    Masa estructural {[}kg{]}           & 0,4                       & 1                         & 2                          \\ \hline
    Masa batería       {[}kg{]}                 & 0,9                       & 1,9                       & 2                          \\ \hline
    Mas electrónica {[}kg{]}            & 0,5                       & 1                         & 1                          \\ \hline
    Empuje restante {[}kgf{]}           & 0,01                      & 0,478                     & 3,116                      \\ \hline
    Tiempo vuelo {[}s{]}                & 252,972973                & 183,5294118               & 131,8309859                \\ \hline
    Precio baterías-propulsor {[}usd{]} & 165,075                   & 365,15                    & 693,005                    \\ \hline
    Costo total {[}ars{]}               & 23968,06463               & 53017,95425               & 101313,866                 \\ \hline
    \end{tabular}
    \caption{Estudio de diferentes EDF's disponibles en el mercado. El requerimiento excluyente para la selección de batería fue que permita vuelo sostenido por 2 minutos.}
    \label{tab:edfseleccion}
\end{table}


El \textit{empuje restante} de la tabla \ref{tab:edfseleccion} es el empuje que sobra luego de restar el peso del vehículo al empuje nominal del EDF. Este empuje dará margen para maniobrar y recuperar la orientación luego de una perturbación externa.

\medskip

Los EDF de aluminio solo se consiguen en el exterior y su precio es en dolares\footnote{El cambio de divisas es desfavorable para el lugar donde se desarrolla este proyecto.}. Esto trae varios inconvenientes en lo referente al presupuesto acotado, logística, compra y obtención. Los EDF
disponibles de plástico tienen una relación de empuje-diámetro mucho menor con respecto a
los de aluminio. Cabe destacar que el costo de un EDF de plástico y un EDF de aluminio son cercanos para diámetros similares.

\medskip

Se decidió por el EDF de 120mm, motor brushless alimentado por una batería Li-Po 12s (48V) 5000mAh 50c/100c para la construcción del prototipo. Este EDF mantiene un empuje resultante positivo para un ángulo de actuación de aproximadamente 45\grad , el cual supera el ángulo actuado máximo de las simulaciones.







\subsection{Diseño de la mecánica}

Para el diseño del gimbal se propuso una distribución de los mecanismos de actuación con
servos concéntricos a los ejes de rotación. Para evitar de esta manera complejidad de
mecanismos, cantidad de piezas de conexión entre servo y ejes, manufactura de mecanismos,
uniones rotoides, y obtener así un mapeo lineal del ángulo de actuación. El único intermediario
es el rodamiento que como se dispuso en la configuración ocupa el mínimo lugar posible justo
por encima de la estría del servo y se lleva las cargas.
Este mecanismo viene con la desventaja de una reducción de la resolución del ángulo de actuación comparado a un sistema actuado por un mecanismo biela-manivela.

\medskip

El material seleccionado para la fabricación del gimbal fue aluminio serie siete mil calidad
aeroespacial. Esta construido de una sola pieza, siendo el elemento que se lleva las tensiones
en todo momento en variabilidad de ángulos hacia el fuselaje, por ello la decisión de ser de la
serie de mayor resistencia mecánica de los aluminios comerciales.

\medskip

Los rodamientos seleccionados son de dimensiones 18mm exterior, 10mm interior y 7mm de espesor,
elegidos de esta manera por una cuestión de espacio para poder pasar la estría de los
servomotores por dentro, dejando un espesor de la pared del estriado de 2.5mm,
para evitar fisuras y hacer posible la manufactura de la pieza. La forma de generar el estriado
interno, es la de indentar con un patrón sobre un agujero previo de diámetro medio a los
valores máximos y mínimos de las crestas y valles de la estría. Esta operación debe hacerse con
el material en bruto para no arruinar las tolerancias que necesitan los rodamientos. Los
rodamientos se montarían clavados, minimizando el peso al no agregar anillos seguers ni tapas con
bulones.

\medskip 

Los ejes del gimbal estarían en disposición simplemente apoyada y de forma axisimétrica para prevenir perturbaciones dinámicas por desbalanceo.

\medskip

Con respecto al fuselaje, al inicio se pensó una envolvente cilíndrica para el anillo externo del
gimbal. Luego se diseñó un desarrollo reticulado optimizado, se pasó por diferentes modelos e
ideas, hasta que se combinaron varios puntos fuertes de cada idea. La envolvente del gimbal se fabrica de forma rolada y optimizada en peso a raíz de
una planchuela de aluminio vaciada y luego generando su forma cilíndrica.\footnote{Contiene al
anillo del gimbal.} Esto distribuye las masas de manera más favorable con menos espacio
ocupado (dinámica-peso). Aguas arriba del gimbal se opto por un chasis tubular, con diversos
vaciados, que mediante flejes puede soportar cada elemento que se acopla al cohete por
medio de uniones abulonadas. Permitiendo mediante sus aberturas el acceso a cada
componente del vehículo, proporcionando, además, una renovación del aire para una
evacuación del calor generado, y un flujo abundante hacia la admisión del EDF.

\medskip

De la manera que se construye el gimbal puede entregar una rotación entera sin hacer contacto con la estructura. Se elige esta configuración por posibles desviaciones del proyecto en el
futuro, calculamos que utilizaremos menos de 20$^\circ$ de rotación de cada eje de gimbal ($\pm$10$^\circ$).

\subsection{Posición de baterías}

La la decisión de donde posicionar las baterías surgía de querer simular un vehículo semejante a los VTVL con motores a reacción y también la posibilidad de tener una inercia favorable para los margenes de estabilidad. Estos dos últimos puntos sugieren que la posición ideal para las baterías es arriba de todo. Esto haría que el punto alrededor del cual se linealizó las ecuaciones de movimiento sea más estable. Luego de una conversación con Pablo Cossutta, un ingeniero electrónico especialista en sistemas de potencia, se optó por la configuración encontrada en el documento. Las baterías se encuentran cerca del EDF para alejar las líneas de potencia trifásica correspondientes al motor brushlesss de lo que es la electrónica digital. Al inestabilizar el punto de operación se obtiene una mejora en la respuesta ante actuaciones permitiendo una corrección de trayectoria más rápida.\footnote{Este resultado es deseado cuando se desea tener mejor rendimiento por ángulo de actuación, como sucede con los \textit{aviones caza} que utilizan este fenómeno a conveniencia.}


\subsection{Selección de servos} \label{ssec:servoSeleccion}

Para obtener una buena respuesta del vehículo ante actuaciones se debe acotar la resolución necesaria. Según \cite{castillo2018efectos}, la resolución angular de un servo analógico está dada por


\[
R_p = \frac{\theta \cdot T_D}{PW}  
\]

donde $\theta$ es el angulo de barrido del servo (especificado por el fabricante), $T_D$ es el tiempo muerto y $PW$ es el ancho de pulso operativo. 

El servo seleccionado es el SC1258TG. Tiene las siguientes especificaciones

\begin{itemize}
    \item Tiempo muerto (\textit{deadband}) 3\micro s
    \item Rango de ancho de pulso mínimo y máximo 800-2100\micro s
    \item Posición neutra 1500\micro s
    \item Ángulo de barrido operativo 100\grad (para 1000-2000\micro s)
    \item Velocidad 1,05 rad/s
    \item Controlador digital
\end{itemize}

Se tiene entonces una resolución mínima de 0,3\grad~ con un ancho de pulso de 2000\micro s. Esta cuenta cede la resolución para un servo analógico, en el caso de tener un servo digital se toma el límite superior entre este valor y la resolución del controlador digital. Como el fabricante no especifica el controlador utilizado, se supone el peor de los casos: un controlador de 8 bits. Este caso tiene una resolución de 0,4\grad~.

Esta resolución es alimentada como parámetro de actuación en las simulaciones.

\subsection{Tren de aterrizaje}
Se definen algunos requerimientos para el tren de aterrizaje
\begin{itemize}
    \item Las patas se dimensionaron para soportar 10 veces el peso del vehículo. Este factor de seguridad toma en cuenta la posibilidad de aterrizaje desparejo y el factor dinámico de un impacto. %\todo{20 veces? Cuantas veces?}
    \item Facilidad de fabricación y armado.
    \item Permitir funcionamiento del vehículo sin interferir con partes móviles.
    \item Permitir flujo libre al EDF.
    \item Buena relación peso-rigidez.
    \item Posibilidad de acople de mecanismo de suspensión.
    \item Mantenga una distancia prudente entre el EDF y la superficie de apoyo para mitigar el efecto suelo.
\end{itemize}

El primer diseño consistía en una estructura reticulada, la cual se decidió abandonar por la dificultad del armado y sucesivas soldaduras, consecuentes alineaciones por las dilataciones térmicas.

\medskip

El diseño seleccionado consiste en unas patas compuestas de una placa de aluminio vaciada y soldada a la estructura principal del fuselaje. Otorga rigidez a expensas de bajo peso y facilidad de construcción.


\subsection{Selección de controlador y sensores}

Se pasó por dos etapas hasta llegar a la decisión final del controlador. Se empezó planteando la utilización de una Raspberry Pi 4B+ para controlar los actuadores con PWM. Esta configuración requeriría el diseño de una placa ad-hoc para desacoplar galvánicamente la Raspberry del controlador de velocidad del EDF (\gls{esc}). Se descartó esta idea a favor de usar una combinación de Raspberry Pi y una placa propia de LIA Aerospace.

\medskip

El \textbf{controlador} a usar es el ARM Cortex-A72 que sería comprado en el paquete comercial (\gls{soc}) conocido como \textit{Raspberry Pi 4B+}. El producto provee salidas para los siguientes usos

\begin{itemize}
    \item \glsxtrshort{uart}
    \item \glsxtrshort{spi}
    \item \glsxtrshort{i2c}
    \item \glsxtrshort{gpio}
\end{itemize}

El controlador ira montado sobre una placa cuyo desarrollo pertenece a LIA Aerospace denominada \textbf{LIA-Board}. Esta placa hace interfaz con el controlador mediante el GPIO header del \gls{soc}. Esto le permite al controlador acceder a los periféricos y salidas disponibles del LIA-Board que controlaran los actuadores y leerán los sensores.

\subsection{Sistema anti rolido}

Para contrarestar el torque generado por la energía rotacional otorgado al flujo de salida se tiene que implementar un sistema \textit{anti-roll}.

\medskip

Se diseño un sistema anti rolido, controlando el movimiento en el eje z del cohete. El mismo
está compuesto de dos servomotores de actuación espejada. Estos están ubicados por debajo del EDF para utilizar la energía del flujo saliente mitigando este efecto. Los álabes son actuados de forma directa por dos servos. Un anillo ensamblado al EDF desarmable hace que el accionar de los álabes se encuentre en una disposición simplemente apoyada. La extensión de las dos piezas de aluminio que son los
soportes del EDF hacen posible la fijación de los micro servos para el comando de los flaps.


\subsection{Contexto de pandemia}
Las medidas tomadas durante la pandemia por el gobierno fueron muy estrictas e influenció a todo tipo de acción que se quiso tomar. Durante el primer año de pandemia no nos pudimos reunir físicamente para discutir ideas de diseño, lo cual dificultó el avance físico del proyecto como así también la toma de decisiones y la comunicación entre partes, crucial en el inicio de todos proyectos de ingeniería.

\medskip

Una de las mayores complicaciones que se tuvo fue la compra de
componentes, además de la fabricación, que se vio afectada, por las políticas cambiantes de
nuestro país con respecto a la compra y la entrada de productos importados a suelo argentino.
Incluso la compra y llegada de los productos fue un motivo de festejo luego de un tortuoso
trayecto.

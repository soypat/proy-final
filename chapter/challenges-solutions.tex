\section{Desafíos y soluciones}\label{sec:challenges-solutions}

En lo que se refiere a la construcción del vehículo, el diseño propuesto paso por varias etapas y
decisiones de ingeniería hasta su forma final. Desde su diseño inicial en lápiz y papel hasta la
creación del CAD con el detalle del ultimo bulón del ensamble para ser cargado al sistema de control en la forma de una matriz de inercia, centro de masa y dinámicas de vuelo, e ir en paralelo retroalimentándose con estos parámetros. En base a esto se realizaron modificaciones sobre el código de control y las simulaciones. A lo largo de este proceso se fueron encontrando desafíos y se fueron implementando soluciones que se detallan en la presente sección.


\subsection{Rediseño de sistema de flaps}

Cuando se estaba realizando la etapa de diseño, se debía realizar la implementación de flaps antirrolido. Estos flaps tienen como misión desviar el flujo proveniente del EDF para obtener como resultado un control de rolido. Se realizó el diseño en CAD en una etapa temprana del proyecto, con la guía de planos del fabricante que se consiguieron por el sitio oficial del fabricante. Estos planos no contenían el nivel de detalle que permitieran conocer que parte rotaba por su superficie externa, para poder realizar el diseño previo de todo el proyecto antes de que se logren importar todas las piezas aledañas. Se prototipo la pieza que contenía los ejes de los flaps simplemente apoyados, para poder ser fabricada con manufactura aditiva en una impresora 3D con PLA. Se procedió a realizar la pieza y tenerla lista a la espera de la importación del EDF. Cuando el EDF y las piezas ingresaron al taller de LIA Aerospace se procedió con el armado. Se detectó que el anillo contenedor del apoyo de los flaps que había sido diseñado anteriormente, apretaba sobre una pieza móvil del EDF, la cual no había sido especificada en los planos. Esto llevó a tener que realizar un rediseño completo de flaps y apoyos, con un diseño mucho más complejo al tener algunas piezas ya fabricadas que imponían condiciones particulares para que todo encaje y tenga un accionar correcto del mecanismo antirrolido.

\medskip

Cuando se probó el sistema antirrolido, los microservos que accionaban el mecanismo de los flaps, dejaron de funcionar, esto sucedió debido a que la caja de engranajes era de plástico y la calidad de este componente no era acorde al tipo de proyecto y utilización que se estaba llevando a cabo. Según los datos del fabricante esta pieza estaba en especificación para la utilización que le estábamos dando, pero las prácticas determinaron que esto no era así. Esta situación llevó a tener que cambiarlos por unos servos con caja de engranajes de metal y salida estriada de metal para el accionar del mecanismo flap antirrolido. Como así también un rediseño de los anclajes de la envuelta de los servomotores.

\subsection{Fallo de programa de vuelo}
Durante una prueba de campo el prototipo entró en un estado degenerado. Esta fue una situación donde el ruido del EDF no permitía la comunicación verbal entre el equipo presente, aún estando a 3 metros de distancia. Se encontraban 2 operadores en el lugar, uno manejando el control del vehículo a través de la interfaz gráfica y otro operador manejando la soga que prevenía que el vehículo vuele más alto de 3 metros como se había convenido.

Comenzó con un momento de confusión cuando luego de comandar a que se detenga el EDF no se detuvo. El operador de la interfaz intentó recuperar el control a través del fail safe pero esto tardaba varios segundos. Mientras tanto el operador de la soga no se enteró de la falla y dejó que el vehículo se acerque al límite de los 3 metros. Al no detenerse el motor como se venía haciendo el vehículo se aceleró bruscamente y al inclinarse comenzó la pérdida total de control de la situación mientras el EDF comenzaba a agregar energía a los movimientos rotacionales del vehículo. Esto último sucedió en menos de 3 segundos. Luego el operador de la soga procedió a realizar una secuencia de apagado forzoso a mano. Para lograr esto había que acercarse al dispositivo, lograr capturarlo y apagarlo manualmente.

Durante esta maniobra una soga se metió por la admisión del EDF provocando la rotura crítica del EDF. El EDF en este momento estaba girando por encima de las 30.000 vueltas por minuto, provocando que se desprendan partes de aluminio de los álabes. De esta forma quedó desbalanceado rotando a muy alta velocidad haciendo un ruido que podría ser descrito como ``ensordecedor". El operador de la soga igual pudo acercarse al dispositivo y desconectar el enchufe de las baterías que estaba en un lugar de difícil acceso. Tuvo que abrazar al dispositivo con los codos y utilizar las manos para desconectar la batería.

Esta situación dejó en evidencia falta de preparación para un ensayo de estas características, con lo que trajo algunas mejoras con respecto a la seguridad:

\begin{itemize}
    \item Una señal visual para cuando las cosas se salen de control y activar un protocolo de emergencia.
    \item Protección completa en los brazos
    \item Una rejilla en la admisión del EDF para que no pueda succionar por error algún objeto.
    \item Un cortacorrientes más accesible para un apagado de emergencia.
    \item Un rediseño para el anclaje en el lugar de ensayo.
    \item Una corrección del bug de software para evitar que el prototipo entre en un estado degenerado en el que no responde a ningún comando, ni siquiera los supremos de Linux.
    \item Agregar un kill switch cableado al PWM que controla el EDF. Si se interrumpe la señal del PWM entonces el EDF se detiene.
\end{itemize}


\subsection{Problemas electricos}
Durante las pruebas fue común observar que al conectar las baterías la computadora de vuelo se apague debido al arc flash que ocurriía adentro del conector. Esto se debía a la alta capacitancia
del \gls{esc} que contiene un banco de capacitores grande. Para subsanar la situación se fabricó un echufe de dos etapas donde el primer contacto permitía cargar los capacitores a traves de un termistor NTC. El termistor tenía una resistencia de miles de ohms nominal que bajaba al momento que comenzaba a circular corriente. Esto permitía cargar los capacitores rápido y de forma segura.

Además del arc flash en varias ocasiones se quemaron salidas digitales optoacopladas de la computadora de vuelo. Resultó aparente que se requerían de componentes de protección adicionales.
Se optó por diseñar la PIAA (sección \ref{ssec:electronicsdesign}) como solución duradera que soportara el entorno hostíl. Para subsanar la situación de forma inmediata se utilizaron salidas digitales que no habían sido quemadas y se les soldaron diodos zener de 9V de protección. Además se puso un optoacoplador secundario para controlar la ESC. Este último optoacoplador iba montado sobre un zócalo que permitía el fácil intercambio ante otra falla.


\subsection{Pruebas del software de estimación de actitud}
Para la evaluación del filtro de Madgwick se programó una visualización \href{https://www.youtube.com/watch?v=M0_s6UW86cs&ab_channel=PatricioWhittingslow}{(Link)} de los resultados del filtro para poder rapidamente evaluar la efectividad del filtro. Se puede ver en el video que el filtro de Madgwick es capaz de estimar la actitud mediante un solo sensor inercial (IMU). El video muestra una revisión del programa preliminar y cabe destacar que el algoritmo recibió varios bugs fixes y mejoras en la implementación. 




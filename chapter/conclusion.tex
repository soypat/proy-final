\section{Conclusión}\label{sec:conclusion}

Como hemos podido apreciar en el presente informe, luego del análisis de cada sistema por separado, el diseño, desarrollo y construcción de cada subconjunto, se pudo lograr la implementación de un software de control escrito enteramente por nosotros adaptado a un vehículo físico, autonomo con tecnlogía VTVL, construido por un equipo de dos estudiantes. Se pudo comprobar el cumplimiento de los requerimientos propuestos al inicio de este proyecto.

% Esto culminó en las pruebas que mostraron que se cumplieron los requerimientos propuestos al inicio del proyecto.

\medskip

El proyecto tuvo dificultades complicadas de sortear debido a que este tipo de tecnologías no es frecuente, ni sencilla, existe solo una empresa en todo el mundo que logró la implementación exitosa a escala mayor y utiliza como medio para completar comercialmente las misiones. No existen antecedentes en instituciónes académicas de libre acceso que permitan una guía o referencia para llevar a cabo un proyecto de similares características. Hasta donde se pudo averiguar no se encontraron casos similares.

% Cada subconjunto podría haber sido un proyecto final, teniendo en cuenta el costo, y el tiempo disponible en un proyecto final de carrera en el ITBA.

Se destacan la utilidad de las herramientas de software que supimos poder dominar durante la carrera, tanto el uso del CAD que permitió facilitar la construcción y caracterización del vehículo en tiempos de pandemia que nos obligó a trabajar separados, y poder llegar dentro de los plazos estipulados con una aproximación de diseño. Que luego fue evolucionando hasta su forma casi final en la etapa temprana del proyecto. Logrando continuar con una construcción pasando por múltiples procesos productivos, que culminaron en la producción de piezas idénticas a las que están en la pantalla. Cuando se comenzaron las tareas de ensamble e integración, las piezas fueron integradas al vehículo sin mayores inconvenientes debido a la precisión manejada por el software. También se menciona la utilización del modelado de elementos finitos, que nos proporcionó la información necesaria para poder verificar una geometría. Es importante destacar que este tipo de herramientas nos permitió la comunicación eficiente con documentación clara y concisa. Teniendo condensado en una imagen o gráfico lo necesario para tomar una decisión de diseño para luego poder tener un éxito en la misión.

\medskip

Se destaca también, que se aprendió que, en los proyectos de ingeniería de tal complejidad, es un acierto invertir tiempo en las decisiones en las etapas tempranas. Diseñando, simulando, discutiendo la funcionalidad e implementación de piezas caras y complejas. Esto permite corregir a tiempo antes que se invierta tiempo y dinero en diseños que no habían nacido correctamente y que por más que se optimicen en un futuro el problema es más profundo a tal punto de determinar que una pieza no es necesaria.

Se pudo lograr entrar dentro del presupuesto diseñando piezas para ser construidas en máquinas tradicionales. Se dejó de lado la utilización de un control numérico de varios ejes. Este proyecto puede ser replicado con un torno y una fresa, además de herramientas manuales cotidianas. Esto se logró pensando la pieza desde varios frentes, no solo de forma funcional, sino como así también las herramientas que había a disposición y el costo asociado a las horas, tan simple como se pueda.

Las pruebas de campo otorgaron resultados positivos, pese al incidente hablado en la sección \ref{sec:challenges-solutions}. Se pudieron sortear los distintos problemas para controlar y realizar vuelos exitosos mediante la implementación del software de vuelo en el dispositivo que se construyó. Produciendo un despegue y aterrizaje exitoso. Esto se logró utilizando en gran parte componentes comercialmente disponibles, esto es a excepción de las piezas custom que teníamos la capacidad de volver a proveer rápidamente si fuera necesario. Se pudieron conseguir nuevamente las partes
que se rompieron para continuar con el presente trabajo. Por esto es que los componentes COTS fueron vitales para poder comprobar una hipótesis y no quedar en el camino demorado o estancado por fallas. Eventualmente el hardware traerá problemas.

El ingeniero aeroespacial a cargo del sistema de control del prototipo FROG de la ESA, Stéphane Querry de Polyvionics, dio su aprobacion del proyecto luego de una conversación con el equipo acerca los algoritmos de control.
Lo que generó el conocimiento acerca del proyecto en Stephane fue de tal magnitud que stephane ofrecio ser mentor en el desarrollo de software. Después de recibir una presentación del proyecto, Stéphane se mostró dispuesto a brindar su experiencia como mentor en el desarrollo del software. Esto fue aprovechado varias veces en la forma de reuniones virtuales donde se hicieron preguntas a Stéphane acerca temas de GNC para trabajos similares en LIA Aerospace. En varias ocasiones Stéphane verbalizó incredulidad amistosamente acerca el hecho que un proyecto como el presente estaba siendo llevado adelante por dos personas.

% Lo que generó el conocimiento acerca del proyecto en Stephane fue tal que el mismo ofrecio ser mentor en el desarrollo de software. 

\medskip

Haber trabajado sobre este proyecto nos permitió crecer en el ambiente aeroespacial y poder solucionar problemas similares en el día a día. Las equivocaciones en el trayecto hasta la finalización del prototipo o la frase “esto podría estar mejor” salieron a la luz para ser solucionadas posteriormente. Es algo no menor en el trabajo de un casi ingeniero trabajando en la mesa de diseño cuando las características del proyecto no admiten equivocaciones.

Para culminar, estamos orgullosos y felices de poder aportar un granito de arena con el desarrollo de comienzo a fin de tecnologías que permitirían la realización de vuelos espaciales suborbitales e incluso orbital con el escalado del prototipo.

Por nuestra parte esperamos poder fomentar y alentar a que se puedan seguir generando documentos e investigaciones sobre este tipo de tecnologías en la institución, siendo que no habían antecedentes, más allá de las limitaciones geo espaciales, económicas, las barreras de importación u opiniones acerca de la complejidad, cuando se trabaja en equipo y con pasión las misiones se pueden lograr.


\section{Trabajo a futuro}

Se detallan puntos que se consideran importantes para continuar con el desarrollo del la tecnología propuesta en el presente trabajo e incluso para poder escalar el proyecto a un nivel mayor.

\begin{itemize}
    \item \textbf{Estimación de actitud:} El estimador usado fue un filtro Madgwick. A futuro sería deseable implementar un filtro de Kalman, que es un estimador de estado óptimo. Esto permitiría obtener una estimación de actitud más robusta ante perturbaciones.
    \item \textbf{Rack de componentes:} Se podría implementar la idea de colocar una pieza contenedora de todos los componentes electrónicos para poder quitarlos y ponerlos de forma rápida a modo de rack. Esto además permitiría realizar pruebas de la electrónica y modificaciones de forma más eficiente.
    \item \textbf{Mejora en la seguridad y fiabilidad:} Se podrían implementar sistemas de seguridad para garantizar un vuelo seguro y confiable. Esto incluiría la integración de redundancias y sistemas de respaldo para mitigar posibles fallos y minimizar el riesgo de accidentes.
    \item \textbf{Integración de tecnologías de machine learning:} La integración de tecnologías de machine learning podrían proporcionar mejoras significativas en el rendimiento del VTVL autónomo. Los sistemas de machine learning podrían ayudar en la toma de decisiones autónomas, el reconocimiento de obstáculos, la planificación de rutas óptimas y la optimización de las maniobras de vuelo.
    \item \textbf{Utilización de sensores de alta calidad:} Los sensores utilizados tienen decadas de uso en la industria. Ya hay sensores de mejor calidad y más precisos que podrían ser utilizados para mejorar la estimación de estado.
    \item \textbf{Sensado absoluto de posición: } Se podría implementar un sistema de sensado absoluto de posición para mejorar la estimación de estado. Esto podría ser un sistema de GPS, un sensor laser funcionando por principio de Time-of-Flight sistema de visión de maquina.
\end{itemize}
\null\newpage
\clearpage

\section{Anexo}

\subsection{Diseño final - imágenes}
\begin{figure}[htb]
    \centering
    \includegraphics[width=\linewidth]{fig/design/v6_3}
    \caption{Prototipo final detalle EDF.}
    \label{fig:design/v6_3}
\end{figure}

\begin{figure}[htb]
    \centering
    \includegraphics[width=\linewidth]{fig/design/v6_5}
    \caption{Prototipo final vista inferior detalle constructivo.}
    \label{fig:design/v6_5}
\end{figure}

\subsection{Diseños preliminares - imágenes}

\begin{figure}[htb]
    \centering
    \includegraphics[width=\linewidth]{fig/design/0}
    \caption{Prototipo de fuselaje con patas reticuladas integro en aluminio, baterías en la nariz, aviónica debajo de la nariz, sin sistema anti rolido.}
    \label{fig:design/0}
\end{figure}


\begin{figure}[htb]
    \centering
    \includegraphics[width=\linewidth]{fig/design/1}
    \caption{Prototipo evolución al fuselaje de aluminio tubular, baterías en la nariz, sistema anti rolido en
    la parte superior, aviónica debajo de las baterías.}
    \label{fig:design/1}
\end{figure}

\begin{figure}[htb]
    \centering
    \includegraphics[width=\linewidth]{fig/design/2}
    \caption{Prototipo fuselaje reticulado de aluminio en forma de placas, baterías en la nariz, sistema anti
    rolido en la parte superior, aviónica debajo de las baterías.}
    \label{fig:design/2}
\end{figure}

\begin{figure}[htb]
    \centering
    \includegraphics[width=\linewidth]{fig/design/3}
    \caption{Prototipo fuselaje reticulado de aluminio en forma de placas, baterías en la nariz, sistema anti
    rolido en la parte superior, aviónica debajo de las baterías.}
    \label{fig:design/3}
\end{figure}

\begin{figure}[htb]
    \centering
    \includegraphics[width=\linewidth]{fig/design/4}
    \caption{Prototipo fuselaje de aluminio tubular vaciado, baterías en el cuerpo, en la parte superior, nariz
    aerodinámica, patas optimizadas, aviónica en la parte inferior, sistema anti-rolido en la parte
    inferior del EDF por derivación del flujo.}
    \label{fig:design/4}
\end{figure}


\begin{figure}[htb]
    \centering
    \includegraphics[width=\linewidth]{fig/design/5}
    \caption{Prototipo fuselaje de aluminio tubular vaciado, baterías en el cuerpo, en el núcleo, nariz aerodinámica, patas optimizadas, aviónica en la parte superior, sistema anti-rolido en la parte
    inferior del EDF por derivación del flujo.}
    \label{fig:design/5}
\end{figure}

\null\newpage
\clearpage

\subsection{Ensamblaje final - imágenes}

\begin{figure}[htb]
    \centering
    \includegraphics[width=\linewidth]{fig/hq/wide_plane_bonete.jpg}
    \caption{Plano general del conjunto armado.}
    \label{fig:hq/wide_plane_bonete}
\end{figure}

\begin{figure}[htb]
    \centering
    \includegraphics[width=\linewidth]{fig/hq/gimbal_close.jpg}
    \caption{Conjunto cardán, servos y soportes de gimbal externo. Note el cableado del gimbal flexible.}
    \label{fig:hq/gimbal_close}
\end{figure}

\begin{figure}[htb]
    \centering
    \includegraphics[width=\linewidth]{fig/hq/electronics.jpg}
    \caption{Conjuntos electrónica y cableado.}
    \label{fig:hq/electronics}
\end{figure}

\begin{figure}[htb]
    \centering
    \includegraphics[width=\linewidth]{fig/hq/flaps.jpg}
    \caption{Vista inferior del conjunto cardán y flaps.}
    \label{fig:hq/flaps}
\end{figure}

\begin{figure}[htb]
    \centering
    \includegraphics[width=\linewidth]{fig/hq/backpcb.jpg}
    \caption{Plano detalle de la placa electrónica de potencia.}
    \label{fig:hq/backpcb}
\end{figure}

\null\newpage
\clearpage
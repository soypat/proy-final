\null\newpage
\clearpage

\section{Anexo}

\subsection{Diseños preliminares - imágenes}

\begin{figure}[htb]
    \centering
    \includegraphics[height=0.25\pdfpageheight]{fig/design/0}
    \caption{Prototipo de fuselaje con patas reticuladas integro en aluminio, baterías en la nariz, aviónica debajo de la nariz, sin sistema anti rolido.}
    \label{fig:design/0}
\end{figure}


\begin{figure}[htb]
    \centering
    \includegraphics[width=\linewidth]{fig/design/1}
    \caption{Prototipo evolución al fuselaje de aluminio tubular, baterías en la nariz, sistema anti rolido en
    la parte superior, aviónica debajo de las baterías.}
    \label{fig:design/1}
\end{figure}

\begin{figure}[htb]
    \centering
    \includegraphics[width=\linewidth]{fig/design/2}
    \caption{Prototipo fuselaje reticulado de aluminio en forma de placas, baterías en la nariz, sistema anti
    rolido en la parte superior, aviónica debajo de las baterías.}
    \label{fig:design/2}
\end{figure}

\begin{figure}[htb]
    \centering
    \includegraphics[width=\linewidth]{fig/design/3}
    \caption{Prototipo fuselaje tubular conico, para favorecer la admision del EDF, patas vaciadas, beterias en la nariz, sistema antirolido en la parte superior,  avionica por debajo de las baterias.}
    \label{fig:design/3}
\end{figure}

\begin{figure}[htb]
    \centering
    \includegraphics[width=\linewidth]{fig/design/4}
    \caption{Un diseño previo al diseño final. Los vaciados son distintos y la disposición de componentes es distinta. La posición de la Raspberry Pi fue cambiada luego de la charla con Pablo Cosutta.}
    \label{fig:design/4}
\end{figure}

\begin{figure}[htb]
    \centering
    \includegraphics[width=\linewidth]{fig/design/5}
    \caption{Prototipo fuselaje de aluminio tubular vaciado, baterías en el cuerpo, en el núcleo, nariz aerodinámica, patas optimizadas, aviónica en la parte superior, sistema anti-rolido en la parte
    inferior del EDF por derivación del flujo.}
    \label{fig:design/5}
\end{figure}

\null\newpage
\clearpage

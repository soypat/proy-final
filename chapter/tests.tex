\section{Pruebas}\label{sec:tests}

\subsection{Objetivo}
Lo que se busca responder son los requerimientos del proyecto establecidos en la sección \ref{sec:scope} de este documento.


\subsection{Metodología}
Se probó el prototipo en las instalaciones de LIA Aerospace. Para esto se colgó de 3 puntos de anclaje con sogas para poder lograr que quede sin estar en contacto con el suelo a una distancia de aproximadamente 1 metro, colgado en el aire, para que en caso de suceder algún desperfecto o salida de control el prototipo no colisione abruptamente con alguna superficie que no sea la indicada, el tren de aterrizaje.



\medskip

En las pruebas lo que se buscaba era lograr una maniobra Hopper, esto es, un despegue y aterrizaje vertical, combinado con un desplazamiento direccional con el vehículo orientado verticalmente.

\medskip

Se iniciaba y se terminaba suspendido en el aire porque el software de vuelo estaba siendo desarrollado aún y los parámetros estaban siendo modificados.
Se agrego además una cuarta soga atada directamente al prototipo para que en caso de una emergencia o salirse de control, tener una posibilidad de guiar el vehículo a mano.


\subsection{Resultados}

Se lograron los siguientes hitos durante las pruebas:
\begin{itemize}
    \item La batería rindió más de 120 segundos de encendido continuo del prototipo, incluyendo el EDF.
    \item El prototipo fue capaz de aterrizar sin daños en todas las pruebas nominales realizadas durante la duración de este proyecto.
    \item El prototipo fue capaz de suspenderse en vuelo y mantenerse en el aire.
    \item El sistema de control respondió ante perturbaciones externas logrando el control del vehículo.
    \item El sistema de control actuó de forma autonoma, sin intervención humana.
\end{itemize}

% Aca seguimos hablando de los resultados para cerrar la seccion:
Las pruebas permitieron poder comprobar que el vehiculo cumplio con los requerimientos propuestos al inicio de este proyecto. Al momento de escribir estas lineas se continuan con las pruebas en paralelo con el desarrollo del software de vuelo, por lo que se espera que en un futuro cercano se logren realizar más ensayos para poder continuar con el desarrollo de esta tecnología.

Adjuntamos un \href{https://youtu.be/0CX27jy0x9w}{link al video} \citep*{pruebasVTVLE2021} donde se muestran fragmentos de las primeras pruebas. 
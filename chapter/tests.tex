\section{Pruebas}\label{sec:tests}

Se probó el prototipo en las instalaciones de LIA Aerospace. Para esto se colgó de 3 puntos de anclaje con sogas para poder lograr que quede sin estar en contacto con el suelo a una distancia de aproximadamente 1 metro, colgado en el aire, para que en caso de suceder algún desperfecto o salida de control el prototipo no colisione abruptamente con alguna superficie que no sea la indicada, el tren de aterrizaje.

\medskip

En las pruebas lo que se buscaba era lograr una maniobra Hopper, esto es, un despegue y aterrizaje vertical, combinado con un desplazamiento direccional con el vehículo orientado verticalmente.

\medskip

Se iniciaba y se terminaba suspendido en el aire porque el software de vuelo estaba siendo desarrollado aún y los parámetros estaban siendo modificados.
Se puso una soga atada directamente al prototipo para que en caso de una emergencia o salirse de control, tener una posibilidad de guiar el vehículo a mano.

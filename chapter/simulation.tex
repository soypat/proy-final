\section{Simulación}

Para comprobar el sistema de control se definió el sistema no-lineal en \Matlab~, se obtuvo el sistema lineal sobre un punto de operación tomando el jacobiano del sistema de ecuaciones diferenciales para controlar el sistema, y finalmente se integró el sistema no lineal en el tiempo retroalimentado con el sistema de control.

Se investigó la respuesta del vehículo ante perturbaciones Delta-Dirac de orientación.

\subsection{Sistema no-lineal}

El sistema \eqref{eq:ssDiffVariables3D} describe la dinámica no-lineal del vehículo con 16 ecuaciones. Estas pueden ser integradas mediante un método numérico para ecuaciones diferenciales ordinarias multivariables no-autónomas. El requerimiento no-autónomo surge de la necesidad de incorporar el vector $\Cu$ a la integración, el cual incluye las actuaciones en base a lo que leen los sensores. 

Para satisfacer el requerimiento no-autónomo se tuvo que implementar un método numérico basado en Runge-Kutta orden 4. El método fue probado y contrastado con soluciones analíticas conocidas.

\subsection{Sistema de control}

Se optó por controlar mediante el controlador \gls{lqr} debido a la simplicidad de implementación y adaptabilidad para problemas de variables de estado. Como se mencionó anteriormente, se obtiene el jacobiano del sistema \eqref{eq:ssDiffVariables3D} alrededor del punto de operación. Esta es la matriz del sistema $\MA$. La matriz $\MB$ también es el jacobiano del sistema pero diferenciado respecto $\Cu$. Finalmente, $\MC$ es la combinación lineal de las mediciones de los sensores (ver sección \ref{sec:model2d} para entender el proceso). 

Se modelaron las siguientes imperfecciones en el sistema:

\begin{itemize}
    \item Delay en medición/actuación
    \item Desalineación de sensores (acelerómetro y giróscopo)
\end{itemize}

La matriz costo asociada al equilibrio es construida asignando los siguientes valores a la diagonal: 5 a las posiciones globales, 1 a las velocidades, 1e-3 a la velocidad del rotor del EDF, 1e-4 a la velocidad angular en pitch y yaw del vehículo, y 1e-5 a las variables restantes (actuadores, ángulos de Euler y velocidad angular en roll).

La matriz costo asociada a los actuadores es diagonal con los siguientes valores: 1000 a actuadores de pitch y yaw del gimbal, 1e-5 al actuador de roll, y 1e-6 al control velocidad del rotor del EDF.


\subsection{Resultados de simulación}


